% preamble.tex
%===============================================================================
% A general preamble to use for most of my documents.  See preambles specific to
%   particular journals or projects under separate headings
%===============================================================================

%===============================================================================
% The following are useful packages that are ALWAYS USED.  See further below
%   for optional packages (and options) that may or maynot be used.
%===============================================================================
%-------------------------------------------------------------------------------
% American Math Society packages that allow use of particular symbols and fonts.
%   For example, these are required to be able to use \eqref{} when referring to
%   a labelled equation.
%-------------------------------------------------------------------------------
\usepackage{amssymb,amsmath,amsfonts}

%-------------------------------------------------------------------------------
% graphicx package controls aspects of graphics.
%-------------------------------------------------------------------------------
\usepackage{graphicx}


%-------------------------------------------------------------------------------
% natbib bibliography package.  This is currently set to use the author,year
%   method rather than the numerical citation method.  In addition, the 
%   parentheses are rounded rather than using square brackets.  Different
%   biobliography styles can also be used -- this is currently set to my afs
%   style.  Finally, the bibliography puncation can be controlled.  This is 
%   currently set to show citations with starting and closing parentheses,
%   a semi-colon between multiple citations, author-year, nothing between author
%   and year, and comma between multiple years of same author.
%-------------------------------------------------------------------------------
\usepackage[authoryear,round]{natbib}
  \bibliographystyle{c:/aaaWork/zGnrlLatex/afs}
  \bibpunct{(}{)}{;}{a}{}{,}
  
%-------------------------------------------------------------------------------
% ccaption package with the option below it is used to replace the colon with a
%   period in table and figure captions.
%-------------------------------------------------------------------------------
\usepackage{ccaption}
  \captiondelim{. }




%===============================================================================
%===============================================================================
% some new commands
%===============================================================================
%===============================================================================
%-------------------------------------------------------------------------------
% Helps make summations, products, and integrals look better -- puts values
%   above and below sigma, pi, or integral rather than behind it.
%   Use \Sum_{i}^{n}
%-------------------------------------------------------------------------------
\newcommand{\Sum}{ 
  \displaystyle\sum                         
}

\newcommand{\Prod}{
  \displaystyle\prod
}

\newcommand{\Int}{
  \displaystyle\int
}

%-------------------------------------------------------------------------------
% Defines figure, table, etc references so you don't have to type Figure each
%   time.  Note that the 'p' version puts the item in parentheses.
%-------------------------------------------------------------------------------
\newcommand{\figref}[1]{%
Figure \ref{#1}%
}

\newcommand{\figrefp}[1]{%
(Figure \ref{#1})%
}

\newcommand{\tabref}[1]{%
Table \ref{#1}%
}

\newcommand{\tabrefp}[1]{%
(Table \ref{#1})%
}

\newcommand{\sectref}[1]{%
Section \ref{#1}%
}

\newcommand{\sectrefp}[1]{%
(Section \ref{#1})%
}

